\section{Prezentacja aplikacji}

\begin{frame}{\insertsection}
\begin{block}{Formularz logowania i rejestracji}
			Aplikacja webowa wymaga autoryzacji.
\end{block}
	\begin{columns}
		\column{0.5\textwidth} 
		\begin{figure}
			\centering
			\includegraphics[width=1\linewidth]{../images/loginang.png}
		\end{figure}
		\column{0.5\textwidth} 
		\begin{figure}
			\centering
			\includegraphics[width=1\linewidth]{../images/Rejestracja}
		\end{figure}
	\end{columns}
\end{frame}

\begin{frame}{\insertsection}
	\begin{block}{Strona główna}
		Informacje o stanie wszystkich kont.	
	\end{block}
	\begin{figure}
		\centering
		\includegraphics[width=0.7\linewidth,keepaspectratio]{../images/HomePage1}
	\end{figure}
\end{frame}
\begin{frame}{\insertsection}
	\begin{block}{Okna modalne do tworzenia kont bankowych i portfeli}
		Użytkownik może otwierać nowe rachunki.
	\end{block}
	\begin{columns}[t] % kolumny wyrównane do góry
		\column{0.49\textwidth}
		% minipage o stałej wysokości (tu 4.5cm) - w środku obraz wycentrowany, podpis poniżej (na stałej pozycji)
		\begin{minipage}[t][4.5cm][c]{\linewidth}
			\centering
			\adjustbox{valign=c}{\includegraphics[width=\linewidth,keepaspectratio]{../images/newAccountModal}}
		\end{minipage}
		\vspace{0.6ex}
		
		\column{0.49\textwidth}
		\begin{minipage}[t][4.5cm][c]{\linewidth}
			\centering
			\adjustbox{valign=c}{\includegraphics[width=\linewidth,keepaspectratio]{../images/newWalletModal}}
		\end{minipage}
		
	\end{columns}
\end{frame}
\begin{frame}{\insertsection}
	\begin{block}{Podsumowanie finansowe wszystkich kont}
		Przedstawienie w formie wykresów danych o transakcjach użytkownika.
	\end{block}
		\begin{figure}
		\centering
		\includegraphics[width=0.7\linewidth,keepaspectratio]{../images/HomePage2}
	\end{figure}
\end{frame}

\begin{frame}{\insertsection}
	\begin{columns}
		\column{0.48\textwidth}
		\begin{minipage}[t][6cm][t]{\linewidth} % ta sama wysokość co po lewej
		\begin{block}{Pasek nawigacyjny}
				Udostępnia możliwość powrotu do strony głównej, dostęp do ustawień oraz wyświetlania danych użytkownika.
		\end{block}
			\vfill
			\centering
			\includegraphics[height=0.67cm, keepaspectratio]{../images/Navbar}
			\vspace{0.6ex}

		\end{minipage}
		\column{0.48\textwidth}
		\begin{minipage}[t][6cm][t]{\linewidth} % ta sama wysokość co po lewej
			\begin{block}{Ustawienia}
				Umożliwia personalizację elementów aplikacji.
			\end{block}
			\vfill
			\centering
			\includegraphics[height=3cm, keepaspectratio]{../images/Ustawienia}
		\end{minipage}
	\end{columns}
\end{frame}
\begin{frame}{\insertsection}
	\begin{figure}
		\centering
		\includegraphics[width=1\linewidth]{../images/polski}
	
	\end{figure}

\end{frame}
\begin{frame}{\insertsection}


	\centering
	\includegraphics[width=1\linewidth]{../images/angielski}
\end{frame}
\begin{frame}{\insertsection}
	\begin{block}{Informacje o użytkowniku}
		Klient może uzupełnić informację o swoim koncie.
	\end{block}
	\begin{figure}
		\centering
		\includegraphics[width=1\linewidth]{../images/UserSettings}
		\label{fig:detalekonta}
	\end{figure}
\end{frame}

\begin{frame}{\insertsection}
	\begin{block}{Szczegóły portfela}
		Portfele pozwalają nam na organizację wydatków z uwzględnieniem kategorii, sprawdzenie salda i różnicy w kwocie na okres czasu, utworzenie transakcji, wyświetlenie, edycję i usuwanie transakcji oraz analizę danych na wykresach.
	\end{block}
\only<1>{
\begin{figure}
	\centering
	\includegraphics[width=1\linewidth, keepaspectratio]{../images/walletDetails}
	\label{fig:walletpage}
\end{figure}
}
\only<2>{
	\begin{columns}
		\column{0.49\linewidth}
		\begin{figure}
			\centering
			\includegraphics[width=1\linewidth]{../images/WalletModal}
			\label{fig:walletmodal}
		\end{figure}
		\column{0.49\linewidth}
		\begin{figure}
			\centering
			\includegraphics[width=0.6\linewidth]{../images/WalletCategory}
			\label{fig:walletcategory}
		\end{figure}	
	\end{columns}
}
\only<3>{
\begin{figure}
	\centering
	\includegraphics[width=0.45\linewidth]{../images/WalletHistory}
	\caption{}
	\label{fig:wallethistory}
\end{figure}

}
\only<4>{
\begin{figure}
	\centering
	\includegraphics[width=0.9\linewidth]{../images/WalletChart1}
	\label{fig:walletchart1}
\end{figure}
}
\only<5>{
\begin{figure}
	\centering
	\includegraphics[width=0.9\linewidth]{../images/WalletChart2}
	\label{fig:walletchart2}
\end{figure}
}
\end{frame}

\begin{frame}{\insertsection}
	\begin{block}{Szczegóły konta bankowego}
		Z poziomu konta bankowego użytkownik widzi informację o koncie, może utworzyć transakcję, sprawdzić historię transakcji oraz przeanalizować wydatki na wykresach.
	\end{block}
	\only<1>{
		\begin{figure}
			\centering
			\includegraphics[width=0.8\linewidth, keepaspectratio]{../images/AccountPage}
			\label{fig:daneportfela}
		\end{figure}
	}
	\only<2>{
		\begin{figure}
			\centering
			\includegraphics[width=0.69\linewidth, keepaspectratio]{../images/AccountCharts}
			\label{fig:daneportfelaA}
		\end{figure}
	}
\end{frame}
