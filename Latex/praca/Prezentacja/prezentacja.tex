\documentclass{beamer}

\usepackage[T1]{fontenc}
\usepackage[polish]{babel}
\usepackage[utf8]{inputenc}
\usepackage{lmodern}
\usepackage{graphicx}
\usepackage{adjustbox} 
\usepackage{caption} 

\selectlanguage{polish}

\usetheme{Boadilla}

\setbeamertemplate{navigation symbols}{}
\setbeamertemplate{footline}{
	\leavevmode%
	\hbox{%
		% Lewa część — 0.25\textwidth
		\begin{beamercolorbox}[wd=0.25\textwidth,ht=2.5ex,dp=1.125ex,center]{author in head/foot}%
			\usebeamerfont{author in head/foot}\hspace*{2ex}\insertshortauthor
		\end{beamercolorbox}%
		% Środkowa część — 0.5\textwidth
		\begin{beamercolorbox}[wd=0.5\textwidth,ht=2.5ex,dp=1.125ex,center]{title in head/foot}%
			\usebeamerfont{title in head/foot}\insertshorttitle
		\end{beamercolorbox}%
		% Prawa część — 0.25\textwidth
		\begin{beamercolorbox}[wd=0.25\textwidth,ht=2.5ex,dp=1.125ex,center]{date in head/foot}%
			\usebeamerfont{date in head/foot}\insertframenumber{} / \inserttotalframenumber\hspace*{2ex}
		\end{beamercolorbox}%
	}%
	\vskip0pt%
}
\title{Aplikacja wspomagająca zarządzanie budżetem}
\author[Dawid Ziora]{Autor: Dawid Ziora\\Promotor: dr inż. Bartosz Kowalczyk}
\date[]{2026}
\institute[]{Politechnika Częstochowska}
\titlegraphic{\includegraphics[width=0.5\linewidth]{../images/logopcz_z_logo_wiisi.png}}

\AtBeginSection[]{
	\begin{frame}[plain]   % plain usuwa nagłówki/stopki (wygląda jak strona tytułowa)
		\centering
		\vfill
		{\Huge\bfseries\insertsection} % nazwa sekcji
		\vfill
	\end{frame}
}

\begin{document}

\maketitle

\begin{frame}{Spis treści}
	\tableofcontents
\end{frame}

\section{Cel i zakres pracy}
\begin{frame}{\insertsection}
\only<1>{
	\begin{block}{Cel pracy}
	Utworzenie aplikacji webowej umożliwiającej monitorowanie przychodów i wydatków, analizę finansów i planowanie oszczędności. 
\end{block}
\begin{figure}
	\centering
	\includegraphics[height=4cm, keepaspectratio]{../images/Cel}
	\label{fig:cel}
\end{figure}

}
\only<2>{
\begin{block}{Zakres pracy}
	Co ma spełniać system końcowy. 
\end{block}
\begin{figure}
	\centering
	\includegraphics[width=0.7\linewidth]{../images/zakrespracy}
	\label{fig:zakrespracy}
\end{figure}

}
\end{frame}
\section{Technologie i narzędzia}
\begin{frame}{Technologie}
		\begin{columns}
		\column{0.5\textwidth}
\begin{figure}
	\centering
	\includegraphics[height=2.5cm,keepaspectratio]{../images/javaLogo}
	\label{fig:sss}
\end{figure}
	\begin{figure}
	\centering
	\includegraphics[height=1.25cm,keepaspectratio]{../images/MavenLogo}
	\label{fig:maven}
\end{figure}
\begin{figure}
	\centering
	\includegraphics[height=1.5cm,keepaspectratio]{../images/spring}
	\label{fig:spring}
\end{figure}
		\column{0.5\textwidth}
\begin{figure}
	\centering
\includegraphics[height=2cm,keepaspectratio]{../images/ReactIcon}
	\label{fig:reac}
\end{figure}
\begin{figure}
	\centering
			\includegraphics[height=2cm,keepaspectratio]{../images/TypeScript}
	\label{fig:reactt}
\end{figure}
			\begin{figure}
	\centering
	\includegraphics[height=2cm,keepaspectratio]{../images/MongoLogo}
	\label{fig:mongologso}
\end{figure}
		\end{columns}
\end{frame}
\begin{frame}{Narzędzia}
	\begin{columns}
		\column{0.25\textwidth}
		\begin{figure}
			\centering
			\includegraphics[height=1cm,keepaspectratio]{../images/DataGrip_icon}
			\label{fig:datagripicon}
		\end{figure}
		
		\begin{figure}
			\centering
			\includegraphics[height=1cm,keepaspectratio]{../images/IntelliJ_IDEA_icon}
			\label{fig:intellijideaicon}
		\end{figure}	
		
		\begin{figure}
			\centering
			\label{fig:dockerlogo}
			\includegraphics[height=1cm,keepaspectratio]{../images/DockerLogo}
		\end{figure}
		
		\begin{figure}
			\centering
			\includegraphics[height=1cm,keepaspectratio]{../images/GitLogo}
			\label{fig:gitlogo}
		\end{figure}
		
		\begin{figure}
			\centering
			\label{fig:junitlogo}
			\includegraphics[height=1cm,keepaspectratio]{../images/JUnitLogo}
		\end{figure}
		
		\column{0.25\textwidth}
		\begin{figure}
			\centering
			\includegraphics[height=1cm,keepaspectratio]{../images/WebStorm_icon}
			\label{fig:webstormicon}
		\end{figure}
		
		\begin{figure}
			\centering
			\includegraphics[height=1cm,keepaspectratio]{../images/PostmanLogo}
			\label{fig:postmanlogo}
		\end{figure}
		
		\begin{figure}
			\centering
			\includegraphics[height=1cm,keepaspectratio]{../images/ViteLogo}
			\label{fig:vitelogo}
		\end{figure}
		
		\begin{figure}
			\centering
			\includegraphics[height=1cm,keepaspectratio]{../images/lombok}
			\label{fig:lombok}
		\end{figure}
		
		\begin{figure}
			\centering
			\includegraphics[height=1cm, keepaspectratio]{../images/AntLogo}
			
			\label{fig:antlogo}
		\end{figure}
	\end{columns}
\end{frame}
%\section{Technologie}
\begin{frame}{\insertsection\space -- Architektura}
	\begin{block}{Architektura klient-serwer}
		Pozwala określić komunikację między klientem, a serwerem.
	\end{block}
	\begin{figure}
		\centering
		\includegraphics[height=6cm,width=0.5\linewidth]{../images/Klient-Serwer}
		\label{fig:klient-serwer}
	\end{figure}
\end{frame}

\begin{frame}{\insertsection\space -- Backend}
	\begin{block}{Java}
		 Język zapewniający ogromne możliwości.
	\end{block}
\begin{figure}
	\centering
	\includegraphics[width=0.9\linewidth]{../images/javaLogo}
	\label{fig:javalogo}
\end{figure}
\end{frame}

\begin{frame}{\insertsection\space -- Backend}
	\begin{block}{Spring Boot}
		Upraszcza tworzenie, konfigurację i uruchamianie aplikacji.
	\end{block}
	\begin{figure}
		\centering
		\includegraphics[width=0.9\linewidth]{../images/springInit}
		\label{fig:springinit}
	\end{figure}
\end{frame}

\begin{frame}{\insertsection\space -- Backend}
	\begin{block}{Maven}
		Zarządzanie projektem.
	\end{block}
	\begin{figure}
		\centering
		\includegraphics[width=0.9\linewidth]{../images/MavenLogo}
		\label{fig:mavenlogo}
	\end{figure}
\end{frame}

\begin{frame}{\insertsection\space -- Frontend}
	\begin{columns}
		\column{0.45\textwidth}
		\begin{minipage}[t][6cm][t]{\linewidth}
			\begin{block}{React}
				Biblioteka do tworzenia aplikacji webowych.
			\end{block}
			\vfill
			\centering
			\includegraphics[height=2.2cm,keepaspectratio]{../images/ReactIcon}
		\end{minipage}
		\column{0.45\textwidth}
	 	\begin{minipage}[t][6cm][t]{\linewidth}
			\begin{block}{Język TypeScript}
				Język umożliwiający statyczne typowanie.
			\end{block}
			\vfill
			\centering
			\includegraphics[height=2.2cm,keepaspectratio]{../images/TypeScript}
		\end{minipage}
	\end{columns}
\end{frame}

\begin{frame}{\insertsection\space -- Baza danych}
	\begin{block}{MongoDB}
		Nierelacyjna baza danych, przechowująca dane w postaci dokumentów
	\end{block}
	\begin{columns}
		\column{0.2\textwidth}
			\begin{figure}
			\centering
			\includegraphics[width=1\linewidth]{../images/MongoLogo}
			\label{fig:mongologo}
		\end{figure}
		\column{0.75\textwidth}
			\begin{figure}
			\centering
			\includegraphics[width=1\linewidth]{../images/MongoData}

			\label{fig:mongodata}
		\end{figure}
	\end{columns}
\end{frame}

%\section{Narzędzia}

\begin{frame}{\insertsection}
		\begin{columns}
			\column{0.25\textwidth}
			\begin{figure}
				\centering
				\includegraphics[height=1cm,keepaspectratio]{../images/DataGrip_icon}
				\label{fig:datagripicon}
			\end{figure}
				
			\begin{figure}
				\centering
				\includegraphics[height=1cm,keepaspectratio]{../images/IntelliJ_IDEA_icon}
				\label{fig:intellijideaicon}
			\end{figure}	
				
			\begin{figure}
				\centering
				\label{fig:dockerlogo}
				\includegraphics[height=1cm,keepaspectratio]{../images/DockerLogo}
			\end{figure}
			
			\begin{figure}
				\centering
				\includegraphics[height=1cm,keepaspectratio]{../images/GitLogo}
				\label{fig:gitlogo}
			\end{figure}
			
			\begin{figure}
					\centering
					\label{fig:junitlogo}
					\includegraphics[height=1cm,keepaspectratio]{../images/JUnitLogo}
			\end{figure}
				
			\column{0.25\textwidth}
			\begin{figure}
				\centering
				\includegraphics[height=1cm,keepaspectratio]{../images/WebStorm_icon}
				\label{fig:webstormicon}
			\end{figure}
			
			\begin{figure}
				\centering
				\includegraphics[height=1cm,keepaspectratio]{../images/PostmanLogo}
				\label{fig:postmanlogo}
			\end{figure}
			
			\begin{figure}
				\centering
				\includegraphics[height=1cm,keepaspectratio]{../images/ViteLogo}
				\label{fig:vitelogo}
			\end{figure}
			
			\begin{figure}
				\centering
				\includegraphics[height=1cm,keepaspectratio]{../images/lombok}
				\label{fig:lombok}
			\end{figure}
			
			\begin{figure}
				\centering
				\includegraphics[height=1cm, keepaspectratio]{../images/AntLogo}
				
				\label{fig:antlogo}
			\end{figure}

		\column{0.5\textwidth}
		\only<1>{\begin{block}{Docker}
			Narzędzie do konteneryzacji.
			\begin{figure}
				\centering
				\includegraphics[width=1\linewidth]{../images/DockerApp}
				\label{fig:dockerapp}
			\end{figure}
		\end{block}
		}
		\only<2>{\begin{block}{Kontrola wersji}
			Repozytoria z możliwością zapisywania zmian w kodzie.
		\begin{figure}
			\centering
		
			\label{fig:gitlogs}
			\includegraphics[width=1\linewidth]{../images/GitLogs}
		\end{figure}
		\end{block}
		}
		\only<3>{
		\begin{block}{Testy funkcjonalne}
		Sprawdzanie działania funkcji
		\begin{figure}
			\centering
			\includegraphics[width=1\linewidth]{../images/Testy}
		
			\label{fig:testy}
		\end{figure}
		\end{block}
		}
		\only<4>{
		\begin{block}{Testy komunikacji}
			Sprawdzanie odpowiedzi na żądania klienta.
		\begin{figure}
			\centering
			\label{fig:postmantest}
			\includegraphics[width=1\linewidth]{../images/PostmanTest}
		\end{figure}
		\end{block}
		}
		\only<5>{
			\begin{block}{Automatyzacja tworzenia kodu}
				Automatyczne tworzenie kodu szablonowego.
					\begin{figure}
					\centering
					\includegraphics[width=1\linewidth]{../images/lombok2}
					\label{fig:lombok2}
				\end{figure}
			\end{block}
		}
		\only<6>{
			\begin{block}{Środowiska programistyczne (IDE)}
				 Ułatwiają pracę programistą.
			\end{block}
		}
	\end{columns}
\end{frame}
\section{Prezentacja aplikacji}

\begin{frame}{\insertsection}
	\only<1>{
		\begin{block}{Formularz logowania}
			Aplikacja webowa wymaga autoryzacji.
		\end{block}
	}
	\only<2>{
	\begin{block}<2>{Tłumaczenie}
		Możliwość presonalizacji języka.
	\end{block}
	}
	\begin{columns}
		\column{0.5\textwidth} 
		\begin{figure}
			\centering
			\includegraphics[width=1\linewidth]{../images/login.png}
		\end{figure}
		\column{0.5\textwidth} 
		\begin{figure}
			\centering
			\includegraphics[width=1\linewidth]{../images/loginang.png}
		\end{figure}
	\end{columns}
\end{frame}

\begin{frame}{\insertsection}
	\begin{columns}
		\column{0.5\textwidth} 
		\begin{block}{Formularz rejestracji}
			Istnieje możliwość utworzenia konta.
		\end{block}
		\column{0.5\textwidth} 
		\begin{figure}
			\centering
			\includegraphics[width=1\linewidth]{../images/Rejestracja}
			\label{fig:rejestracja}
		\end{figure}
	\end{columns}
\end{frame}

\begin{frame}{\insertsection}
	\begin{block}{Strona główna}
		Umożliwia sprawdzanie informacji o kontach	
	\end{block}
	\begin{figure}
		\centering
		\includegraphics[width=0.9\linewidth]{../images/MotywCiemny}
	\end{figure}
\end{frame}

\begin{frame}{\insertsection}
	\begin{block}{Wybór motywu}
		Personalizacja barw aplikacji.
	\end{block}
		\begin{figure}
		\centering
		\includegraphics[width=0.9\linewidth]{../images/MotywJasny}
	\end{figure}
\end{frame}

\begin{frame}{\insertsection}
	\begin{columns}
		\column{0.48\textwidth}
		\begin{minipage}[t][6cm][t]{\linewidth} % ta sama wysokość co po lewej
		\begin{block}{Pasek nawigacyjny}
				Udostępnia możliwość powrotu do strony głównej, dostęp do ustawień oraz wyświetlania danych użytkownika.
		\end{block}
			\vfill
			\centering
			\includegraphics[height=0.75cm, keepaspectratio]{../images/Navbar}
			\vspace{0.6ex}

		\end{minipage}
		\column{0.48\textwidth}
		\begin{minipage}[t][6cm][t]{\linewidth} % ta sama wysokość co po lewej
			\begin{block}{Ustawienia}
				Pozwala na wybór języka i motywu.
			\end{block}
			\vfill
			\centering
			\includegraphics[height=3cm, keepaspectratio]{../images/Ustawienia}
		\end{minipage}
	\end{columns}
\end{frame}

\begin{frame}{\insertsection}
	\begin{block}{Detale konta}
		Udostępnia informacje o użytkowniku
	\end{block}
	\begin{figure}
		\centering
		\includegraphics[width=0.7\linewidth]{../images/DetaleKonta}
		\label{fig:detalekonta}
	\end{figure}
\end{frame}

\begin{frame}{\insertsection}
\begin{columns}[t] % kolumny wyrównane do góry
	\column{0.33\textwidth}
	% minipage o stałej wysokości (tu 4.5cm) - w środku obraz wycentrowany, podpis poniżej (na stałej pozycji)
	\begin{minipage}[t][4.5cm][c]{\linewidth}
		\centering
		\adjustbox{valign=c}{\includegraphics[width=\linewidth,keepaspectratio]{../images/AccountForm}}
	\end{minipage}
	\vspace{0.6ex}
	\captionof{figure}{Dodawanie konta}
	
	\column{0.33\textwidth}
	\begin{minipage}[t][4.5cm][c]{\linewidth}
		\centering
		\adjustbox{valign=c}{\includegraphics[width=\linewidth,keepaspectratio]{../images/DodawanieUzytkownika}}
	\end{minipage}
	\vspace{0.6ex}
	\captionof{figure}{Dodawanie użytkownika}
	
	\column{0.33\textwidth}
	\begin{minipage}[t][4.5cm][c]{\linewidth}
		\centering
		\adjustbox{valign=c}{\includegraphics[width=\linewidth,keepaspectratio]{../images/UsuwanieUzytkownika}}
	\end{minipage}
	\vspace{0.6ex}
	\captionof{figure}{Usuwanie użytkownika}
\end{columns}
\end{frame}

\begin{frame}{\insertsection}
	\begin{block}{Detale portfela}
		Wyświetlanie danych o wpłatach i wypłatach.
	\end{block}
	\begin{columns}
		\column{0.45\textwidth}
			\begin{figure}
			\centering
			\includegraphics[width=1\linewidth]{../images/DanePortfela}
			\label{fig:daneportfela}
		\end{figure}
		\column{0.45\textwidth}
		\begin{figure}
			\centering
			\includegraphics[width=0.9\linewidth]{../images/TransakcjeHistoria}
			\label{fig:transakcjehistoria}
		\end{figure}
		
	\end{columns}

\end{frame}

\begin{frame}{\insertsection}
\begin{figure}
	\centering
	\includegraphics[width=1\linewidth]{../images/TransakcjeAll}
	\caption{Wszystkie wpłaty i wypłaty}
	\label{fig:transakcjeall}
\end{figure}
\end{frame}

\begin{frame}{\insertsection}
\begin{figure}
	\centering
	\includegraphics[width=1\linewidth]{../images/TransakcjeDaily}
	\caption{Wpłaty i wypłaty za poszczególne dni}
	\label{fig:transakcjedaily}
\end{figure}
\end{frame}

\section{Podsumowanie}
\begin{frame}{Wnioski i osiągnięcia}
	\begin{block}{Wnioski}
		\begin{itemize}
			\item Nowoczesne technologie umożliwiają łatwy rozwój aplikacji.
			\item Przygotowanie wymagań dla systemu pozwala usystematyzować pracę.
			\item Nierelacyjne bazy danych pozwalają na swobodę tworzenia struktury kolekcji.
			\item Testowanie oprogramowania ogranicza ilość błędów.
			\item Dobór odpowiednich technologi przekłada się na wydajność systemu.
		\end{itemize}
	\end{block}
\end{frame}

\end{document}