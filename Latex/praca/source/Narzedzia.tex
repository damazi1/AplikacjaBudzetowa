\chapter{Narzędzia i metodyka pracy}
W rozdziale przedstawiono zestaw narzędzi oraz metodykę pracy zastosowaną podczas tworzenia aplikacji webowej. Opisane rozwiązania pozwoliły na efektywne tworzenie, testowanie i wdrażanie aplikacji.
\section{TODO: Metodyka pracy}
\textbf{TODO: Jaka metodyka np. Zwinna (Agile/Scrum), opisać jak się używa, sprinty, taski itd.}
\section{Narzędzia stosowane w procesie wytwarzania aplikacji webowej}
W procesie tworzenia aplikacji wykorzystuje się narzędzia, które usprawniają pracę programisty. Do takich narzędzi należą zintegrowane środowiska programistyczne (ang. \textit{Integrated Development Enviroment} - IDE), systemy kontroli wersji, narzędzia wspomagające testowanie oraz konteneryzacje.
\subsection*{Środowiska programistyczne}
Każdy deweloper potrzebuje narzędzi, które pozwolą mu na edycję kodu źródłowego. Jednym z najpopularniejszych rozwiązań są zintegrowane środowiska programistyczne, które składają się z edytora kodu, kompilatora lub interpretera oraz debuggera. W projekcie zostały wykorzystane takie edytory jak: Intellij IDEA Ultimate oraz WebStorm. Oba te narzędzia zostały wydane przez producenta oprogramowania JetBrains. 
\subsection*{System kontroli wersji}
System kontroli wersji to jedna z najważniejszych rzeczy podczas tworzenia dowolnej aplikacji. Oprogramowanie służące do obsługi kontroli wersji nazywa się Git. Pozwala na tworzenie lokalnych lub zdalnych repozytoriów do których zapisywane są zmiany w projekcie poprzez polecenie \texttt{commit}. W dowolnym momencie możemy śledzić modyfikacje plików lub powrócić do wcześniejszej wersji projektu przy pomocy komendy \texttt{rollback}. Aby przesłać zmiany do zdalnego repozytorium na stronie trzeciej (np.: github, bitbucket), należy wykorzystać opcję \texttt{push}, natomiast pobranie wymaga operacji \texttt{pull}. 

Git umożliwia również pracę w grupach dzięki systemowi tworzenia gałęzi (ang. \textit{branches}), które pozwalają na równoległą prace całego zespołu. Zakończone części kodu mogą być scalone poleceniem \texttt{marge} z główną gałęzią.
\subsection*{Testowanie komunikacji}
Poprawne komunikowanie się wszystkich warstw jest kluczowym aspektem tworzenia aplikacji webowych. W celu sprawdzenia komunikacji wykorzystuje się narzędzie Postman, które umożliwia wysyłanie żądań HTTP do serwera i sprawdzanie otrzymanych odpowiedzi. Pozwala to na weryfikację punktów końcowych aplikacji. Narzędzie to daje możliwość tworzenia kolekcji z zapytaniami, które można używać wielokrotnie bez konieczności ponownego tworzenia zapytań wraz z danymi. Posiada również możliwość zapisywania ciasteczek.

\subsection*{Konteneryzacja}
Możliwość pracy na różnych urządzeniach bez konieczności instalowania oprogramowania jest ciekawą perspektywą dla każdego dewelopera. Umożliwia to narzędzie Docker, którego głównym elementem jest silnik (ang. \textit{Docker Engine}) działający w formie usługi. Pozwala na budowanie, uruchamianie i zarządzanie kontenerami. Środowisko umożliwia tworzenie obrazów na bazie których powstają izolowane środowiska do wykonywania konkretnych zadań. Narzędzie pozwala na konteneryzacje zaawansowanych projektów, korzystające z wielu usług, dzięki plikowi \texttt{docker-compose.yml}, w którym definiuje się zależności między kontenerami, konfigurację sieci oraz woluminy.

\subsection*{TODO: Testy funkcyjne}
Testy jednostkowe są wykonywane przy pomocy biblioteki JUnit w wersji 5. Testy są wykonywane przed uruchomieniem aplikacji, aby sprawdzić jej poprawne działanie. 
\textbf{TODO: Tu należy dopisać zdecydowanie więcej. Zostanie to uzupełnione w momencie rozszerzenia mojej wiedzy w temacie testów jednostkowych i integracyjnych.}
\subsection*{TODO: Lombok i devTools Spring}
\textbf{TODO: Zależności które opisałem w sekcji SpringBoot teoretycznie są narzędziami. Czy należy je przenieść do sekcji narzędziowej?}
