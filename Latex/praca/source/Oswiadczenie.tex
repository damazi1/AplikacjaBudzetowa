% Definicja komendy do generowania linii do wypełnienia
\newcommand{\liniawypelnienia}[1]{%
	\makebox[0.9\linewidth][l]{#1: \dotfill}%
}
% Formatowanie daty i miejsca
\begin{flushright}
	Częstochowa, dn. \dotfill
\end{flushright}
\vspace{1.5cm}
% Sekcja danych studenta
\begin{tabular}{@{}l@{}}
	\liniawypelnienia{Imię i nazwisko} \\
	\liniawypelnienia{Nr albumu} \\
	\liniawypelnienia{Kierunek} \\
	\liniawypelnienia{Wydział}
\end{tabular}

\vspace{2cm}

% Tytuł Oświadczenia
\begin{center}
	\textbf{\Large Oświadczenie autora pracy dyplomowej*}
\end{center}

\vspace{1cm}

% Treść Oświadczenia
Oświadczam pod rygorem odpowiedzialności karnej, że złożona przeze mnie praca dyplomowa pt.
\dotfill\newline
.\dotfill

\noindent jest moim samodzielnym opracowaniem i nie zawiera treści uzyskanych w sposób niezgodny z obowiązującymi przepisami.

\vspace{0.5cm}

\noindent Jednocześnie oświadczam, że moja praca (w całości ani we fragmentach) nie była wcześniej przedmiotem procedur związanych z uzyskaniem tytułu zawodowego w Politechnice Częstochowskiej.

\vspace{0.5cm}

\noindent Wyrażam/nie wyrażam$^{**}$ zgody/zgody na nieodpłatne wykorzystanie przez Politechnikę Częstochowską całości lub fragmentów wyżej wymienionej pracy w publikacjach Politechniki Częstochowskiej.

\vspace{1cm}

% Sekcja na podpis
\begin{flushright}
	\dotfill \\
	czytelny podpis studenta
\end{flushright}

\vspace{1cm}

\noindent \hrulefill % Pozioma linia

\vspace{0.5cm}

% Uwag
\noindent $^{*}$W przypadku zbiorowej pracy dyplomowej dołącza się Oświadczenie każdego ze współautorów.

\noindent $^{**}$Niepotrzebne skreślić.