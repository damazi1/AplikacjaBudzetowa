\section{React}
React to jedna z bibliotek JavaScript służąca do tworzenia interfejsów użytkownika (ang. User Interface) w aplikacjach webowych. Jest oparta na koncepcji komponentów, które można łatwo ze sobą łączyć i w czytelny sposób wyświetlać na stronie. React pozwala renderować widoki na podstawie mechanizmu Virtual DOM (ang. Document Object Model). Obiekt DOM umożliwia dostęp do struktury strony w celu jej modyfikacji. W przypadku modelu wirtualnego minimalizuje operacje na drzewie rzeczywistym. 
\subsection*{Vite}
W celu usprawnienia procesu tworzenia aplikacji zastosowano narzędzie Vite, które pełni rolę bundlera, czyli łączy ze sobą wiele plików m.in. kody źródłowe i zależności. Rozwiązanie to oferuje szybkie uruchamianie środowiska, optymalizacja kodu i co najważniejsze w usprawnieniu pracy przeładowanie kodu na bieżąco (ang. Hot Module Replacement), który pozwala wyświetlać zmiany bez konieczności ponownego budowania całej aplikacji. 
\subsection*{TypeScript}
Komponenty oraz wszystkie składowe projektu są przygotowane w języku TypeScript, który jest nadzbiorem języka JavaScript. Wprowadza on statyczne typowanie, znaczy to że typy są określone i sprawdzane w czasie kompilacji. Pozwala wprowadzać typy dla zmiennych, funkcji i obiektów. Dzięki temu kod jest bardziej czytelny i zrozumiały dla developera. 
\newline

Połączenie Reacta, Vite oraz TypeScript zapewnia wydajne i przyjemne środowisko dla pracy programisty, lepszą kontrole nad danymi i wydajność pracy. Technologie te zostały wybrane również dla bogatego wyboru bibliotek wspieranych przez react. Jedną z nich jest biblioteka AntD, która zawiera wstępnie przygotowane do użycia na stronie komponenty. 