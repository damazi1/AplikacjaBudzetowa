\chapter{Prezentacja aplikacji}
Celem tego rozdziału jest zaprezentowanie funkcjonalności w~warstwie prezentacji. Interfejs użytkownika składa się z~elementów, które pozwalają w~płynny sposób poruszać się po aplikacji oraz dostosowywać ją do własnych potrzeb. 
\section{Środowisko deweloperskie}
Aplikacja webowa uruchamiana jest w~środowisku deweloperskim. Urządzenie, na którym uruchomiona jest aplikacja, działa jak serwer i~klient jednocześnie. Odwołanie do lokalnego hosta (\textit{localhost}) jest wymagane, by uruchomić aplikację, która znajduje się domyślnie pod adresem \texttt{localhost:5173/}, gdzie liczba po dwukropku oznacza port, z~którego korzysta aplikacja. W~dalszej części rozdziału adres zostanie zastąpiony słowem ,,\texttt{HOST}''.

\section{Formularze autoryzacji}
Po pierwszym załadowaniu aplikacji www użytkownik zostanie przekierowany przez system do formularza logowania znajdującego się pod adresem \texttt{HOST/auth/login}. Zawiera on dwa pola wymagane, w~których użytkownik powinien podać nazwę użytkownika oraz hasło, i~pole wyboru umożliwiające zapisanie danych (rys. \ref{fig:login}). Użytkownik, który nie posiada konta, ma możliwość utworzenia go poprzez kliknięcie odnośnika \textit{Zarejestruj się!}, który przenosi do formularza rejestracji.
\begin{figure}[H]
\begin{minipage}{0.4\textwidth}
	\centering
	\includegraphics[width=\linewidth]{images/Login}
\end{minipage}
\hfill
\begin{minipage}{0.4\textwidth}
	\centering
	\includegraphics[width=\linewidth]{images/LoginAng}
\end{minipage}
	\caption{Formularz logowania po polsku i~angielsku}
	\label{fig:login}
\end{figure}
\begin{figure}[H]
	\begin{minipage}{0.5\textwidth}
	\centering
	\includegraphics[width=\linewidth]{images/Rejestracja}
	\caption{Formularz rejestracji}
	\end{minipage}
\hfill
	\begin{minipage}{0.5\textwidth}
	\centering
	\includegraphics[width=\linewidth]{images/Navbar}
	\caption{Pasek nawigacyjny}
	\label{fig:navbar}
\end{minipage}
\end{figure}

\section{Ustawienia}
Aplikacja webowa oferuje możliwość personalizacji strony poprzez zakładkę ustawień (rys. \ref{fig:ustawienia}) znajdującą się po prawej stronie na pasku nawigacyjnym (ang. \textit{navbar}; rys. \ref{fig:navbar}). Aplikacja pozwala na ustawienie ciemnego oraz jasnego wygląd interfejsu użytkownika (rys. \ref{fig:motywJasny}, \ref{fig:motywCiemny}) oraz zmianę języka na polski lub angielski (rys. \ref{fig:login}).

\begin{figure}[H]

	\centering
	\includegraphics[width=0.9\linewidth]{images/MotywJasny}
	\caption{Motyw Jasny}
	\label{fig:motywJasny}
\end{figure}
\begin{figure}[H]
	\centering
	\includegraphics[width=0.9\linewidth]{images/MotywCiemny}
	\caption{Motywy Ciemny}
	\label{fig:motywCiemny}
\end{figure}

\begin{figure}[H]
	\centering
	\includegraphics[width=0.7\linewidth]{images/Ustawienia}
	\caption{Ustawienia użytkownika}
	\label{fig:ustawienia}
\end{figure}
\section{Funkcjonalność aplikacji udostępniona użytkownikowi zalogowanemu}
Pomyślna próba logowania spowoduje przekierowanie do strony głównej (rys. \ref{fig:homepage1}, \ref{fig:homepage2}). 
Wyświetlone zostaną wszystkie portfele, konta bankowe i~podsumowanie danych na wykresach. Z~poziomu strony głównej można również utworzyć nowy portfel lub dodać konto bankowe (rys. \ref{fig:newwalletmodal}, \ref{fig:newaccountmodal}). 

% TODO: \usepackage{graphicx} required
\begin{figure}[H]
	\centering
	\includegraphics[width=\linewidth]{images/HomePage1}
	\caption{Portfele i~konta}
	\label{fig:homepage1}
\end{figure}

% TODO: \usepackage{graphicx} required
\begin{figure}[H]
	\centering
	\includegraphics[width=\linewidth]{images/HomePage2}
	\caption{Podsumowanie danych użytkownika}
	\label{fig:homepage2}
\end{figure}


\begin{figure}[H]
	\begin{minipage}{0.5\textwidth}
	\centering
	\includegraphics[width=\linewidth]{images/newWalletModal}
	\caption{Formularz utworzenia portfela}
	\label{fig:newwalletmodal}
\end{minipage}
	\begin{minipage}{0.5\textwidth}
	\centering
	\includegraphics[width=\linewidth]{images/newAccountModal}
	\caption{Formularz utworzenia konta bankowego}
	\label{fig:newaccountmodal}
\end{minipage}
\end{figure}

\subsection*{Ustawienia konta użytkownika}
Klient ma możliwość dodania informacji kontaktowych do konta. Podanie danych jest opcjonalne. 

\begin{figure}[H]
	\centering
	\includegraphics[width=\linewidth]{images/UserSettings}
	\caption{Dane kontaktowe użytkownika}
	\label{fig:usersettings}
\end{figure}

\subsection*{Portfel użytkownika}
Klient zarządza swoim portfelem poprzez panel portfelu. Aplikacja wyświetla takie informacje jak nazwa portfel, waluta, balans konta  i~podsumowanie okresowe (rys. \ref{fig:walletdetails}), formularze do dodawania przychodów oraz wydatków konta (rys. \ref{fig:wallettransaction}), historię transakcji (rys. \ref{fig:wallethistory})  i~analizę finansową na wykresach liniowych, słupkowych i~kołowych (rys. \ref{fig:walletoverview}). Informacje są domyślnie wyświetlane z~bieżącego miesiąca, jednak użytkownik może dowolnie dostosować okres, z~którego chce przeprowadzić analizę wydatków. Każda transakcja posiada kategorię, która została nadana przez klienta.  

% TODO: \usepackage{graphicx} required
\begin{figure}[H]
	\centering
	\includegraphics[width=\linewidth]{images/WalletDetails}
	\caption{Główny panel informacyjny}
	\label{fig:walletdetails}
\end{figure}

\begin{figure}[H]
\begin{minipage}{0.5\textwidth}
	\centering
	\includegraphics[width=\linewidth]{images/WalletTransaction}
	\caption{formularz dodawania transakcji}
	\label{fig:wallettransaction}
\end{minipage}
\begin{minipage}{0.5\textwidth}
	\centering
	\includegraphics[width=0.9\linewidth]{images/WalletCategory}
	\caption{Dostępne kategorie}
	\label{fig:walletcategory}
\end{minipage}
\end{figure}


% TODO: \usepackage{graphicx} required
\begin{figure}[H]
	\centering
	\includegraphics[width=0.9\linewidth]{images/WalletHistory}
	\caption{Przykładowa historia transakcji}
	\label{fig:wallethistory}
\end{figure}

% TODO: \usepackage{graphicx} required
\begin{figure}[H]
	\centering
	\includegraphics[width=0.9\linewidth]{images/WalletOverview}
	\caption{Podsumowanie w~formie wykresów}
	\label{fig:walletoverview}
\end{figure}

\subsection*{Konta bankowe}
Strona konta bankowego różni się od portfeli brakiem możliwości kategoryzowania transakcji. Posiada natomiast możliwość tworzenia przelewów. Do utworzenia przelewu wymagana jest znajomość numeru konta odbiorcy. 
% TODO: \usepackage{graphicx} required
\begin{figure}[H]
	\centering
	\includegraphics[width=0.9\linewidth]{images/AccountPage}
	\caption{Detale konta bankowego}
	\label{fig:accountpagea}
\end{figure}
