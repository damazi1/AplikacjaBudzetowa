\chapter{Implementacja systemu}
Rozdział  przedstawia proces implementacji systemu. Omówiono w nim wykorzystane technologie oraz poszczególne warstwy aplikacji. Celem tego rozdziału jest zaprezentowanie działania wykorzystanej architektury oraz korelacje między poszczególnymi elementami systemu.
\section{Backend – logika biznesowa i API}
Warstwa serwerowa odpowiada ze przetwarzanie żądań od warstwy prezentacji oraz za komunikację z bazą danych. Elementy z których jest skonstruowana to m.in. kontrolery, repozytoria, czy pliki konfiguracyjne. Została zrealizowana w oparciu o framework Spring Boot, który upraszcza proces tworzenia aplikacji dzięki zestawowi tzw. starterów.

\subsection*{Spring Boot i ekosystem bibliotek}
Freamework spring boot oferuje moduły oraz biblioteki wykorzystane do budowy REST API, dostępu do baz danych, bezpieczeństwa i testowania. Wykorzystywane biblioteki są deklarowane w pliku \texttt{pom.xml}. Są one wstępnie skonfigurowane. Takie rozwiązanie znacząco zwiększa tępo rozwoju aplikacji, pozwala na lepszą pracę w grupach oraz redukuje ilość kodu szablonowego (ang. \textit{boilerplate code}).
\subsubsection*{Zarządzanie zależnościami i startery}
Spring boot łączy biblioteki w postaci starterów. Pozwala to zminimalizować ryzyko konfliktów wersji oraz skrócić czas konfiguracji. Kluczowe startery wykorzystywane w aplikacji webowej to:
\begin{itemize}
	\item \textbf{spring-boot-starter-web} -- Warstwa HTTP/REST oparta na spring MVC, która udostępnia wbudowany serwer Tomcat. 
	\item \textbf{spring-boot-starter-data-jpa} -- Umożliwia dostęp do relacyjnej bazy danych poprzez JPA (ang. \textit{Java Persistence API}).
	\item \textbf{spring-boot-starter-data-mongodb} -- Pozwala na dostęp do nierelacyjnej bazy MongoDB. 
	\item \textbf{spring-boot-starter-validation} -- Implementuje biblioteki do walidacji danych wejściowych.
	\item \textbf{spring-boot-starter-security} -- Mechanizm uwierzytelniania, autoryzacji i ochrony endpointów.
	\item \textbf{spring-boot-starter-test} -- Środowisko testowe do testów jednostkowych i integracyjnych. 
	\item \textbf{Lombok} -- Redukuje kod szablonowy (gettery, settery, konstruktory).
\end{itemize}
\subsubsection*{Warstwa webowa}
Umożliwia komunikację z wykorzystywaniem standardu REST API (ang. \textit{Representational State Transfer Application Programing Interface}). W warstwie kontrolerów tworzone są endpointy w oparciu o modele z jakimi są związane. Kontroler może implementować obsługę żądań HTTP, takich jak:
\begin{itemize}
	\item POST -- Przesyłanie danych na serwer
	\item GET -- Pobieranie danych z serwera
	\item PUT -- Aktualizowanie danych na serwerze
	\item DELETE -- Usuwanie danych z serwera
\end{itemize}
Mapowanie endpointów w kontrolerze jest pokazane w listingu \ref{lst:java-endpoints}.
\begin{lstlisting}[language={Java}, caption={Przykładowe Endpointy}, label={lst:java-endpoints}]
@RestController 
@RequestMapping("/User") // Wszystkie endpointy zaczynają się od /User
	@PostMapping("/register")		// Dostęp pod POST /User/register
	@GetMapping("/search")			// Dostęp pod GET /User/search
	@PutMapping("/update/{id}")		// Dostęp pod PUT /User/update/{id}
	@DeleteMapping("/delete/{id}")	// Dostęp pod DELETE /User/delete{id}
\end{lstlisting}

Implementacja powyższych endpointów znajduje się w listingach (\ref{lst:java-register}, \ref{lst:java-list}, \ref{lst:java-delete}, \ref{lst:java-update})
\subsubsection*{Dostęp do danych}
Połączenie z bazą danych zostało zrealizowane z wykorzystaniem \\ \texttt{spring-boot-starter-data-mongodb}, który udostępnia spójny model pracy z dokumentową bazą MongoDB. Pozwala mapować obiekty klas na dokumenty i tworzyć repozytoria oparte na interfejsach.

Utworzenie klasy mapowanej do dokumentu wymaga oznaczenia adnotacją \texttt{@Document}, a klucza głównego \texttt{@Id}. Identyfikator przeważnie określa się typem String (listing \ref{lst:java-Account}). 

Warstwa repozytoriów opiera się na interfejsach rozszerzających \texttt{MongoRepository<T, ID>}. Repozytoria pozwalają na podstawie nazw generować implementację metod wyszukujących (listing \ref{lst:java-Repository}).
\begin{lstlisting}[language={Java}, caption={Fragment modelu Account}, label={lst:java-Account}]
@Document
public class Account {
	@Id
	private String id;
	private String name;
	@Size(min = 25, max = 25)
	@Indexed(unique = true)
	private String number;
	private Currency currency;
	private String userId;
	...
\end{lstlisting}
\begin{lstlisting}[language={Java}, caption={Repozytorium dla klasy Account}, label={lst:java-Repository}]
@RepositoryRestResource
public interface AccountRepository extends MongoRepository<Account, String> {
	List<Account> findByUserId(String userId);
	Optional<Account> findByNumber(String number);
}
	\end{lstlisting}
\subsubsection*{Bezpieczeństwo}
Spring Security zapewnia mechanizm uwierzytelniania JWT (ang. \textit{JSON Web Token}) i filtrowania żądań. Token umożliwia autoryzowane połączenie z serwerem dzięki mechanizmowi ciasteczek. Token bezpieczeństwa składa się z trzech części (tab. \ref{tab:JWT}), które zwyczajowo wyglądają w następujący sposób \texttt{xxxxx.yyyyy.zzzzz}. Filtrowanie pozwala m.in. na określenie żądań, które mogą zostać wykonane bez konieczności autoryzacji użytkownika.

\begin{longtable}{|p{3cm}|p{5cm}|p{6cm}|}
		\caption{Struktura tokenu JWT \cite{JWT}}
		\label{tab:JWT} \\
\hline
\textbf{Nawza} & \textbf{struktura} & \textbf{Opis} \\ \hline

Nagłówek (ang. \textit{Header}) &
 \texttt{\{} \newline
 \hspace*{1em}\texttt{"alg": "HS256",} \newline
 \hspace*{1em}\texttt{"typ": "JWT"} \newline
 \texttt{\}} & 
 Nagłówek składa się z wykorzystanego algorytmu oraz z typu użytego tokena. \\ \hline
 Ładunek (ang. \textit{Payload}) &
  \texttt{\{} \newline
 \hspace*{1em}\texttt{"sub": "1234567890",} \newline
 \hspace*{1em}\texttt{"name": "John Doe"} \newline
 \hspace*{1em}\texttt{"admin": "true"} \newline
 \texttt{\}} & 
 Ładunek zawiera roszczenia (ang. \textit{claims}), czyli oświadczenia dotyczące podmiotu i dodatkowe dane. Typy roszczeń to: rejestrowane (ang. \textit{registered}), publiczne (ang. \textit{public}) i prywatne (ang. \textit{private}).
 \\ \hline
  Podpis (ang. \textit{Signature}) &
\texttt{HMACSHA256(} \newline
 \hspace*{1em}\texttt{base64UrlEncode(header)} \newline
 \hspace*{1em}\texttt{+ "."\space +} \newline
 \hspace*{1em}\texttt{base64UrlEncode(payload)} \newline
 \hspace*{1em}\texttt{, secretPassword)} &
  Podpis składa się z zaszyfrowanego nagłówka, ładunku i hasła, które są później przetwarzane przez wybrany algorytm. 
 \\ \hline

\end{longtable}

\subsubsection*{Testowanie}
Narzędzia do wykonywania testów dostarcza \texttt{spring-boot-starter-test}. Testy jednostkowe wykonuje się z wykorzystaniem biblioteki JUnit. Polegają na wysłaniu żądania HTTP do serwera i porównaniu odpowiedzi otrzymanej z oczekiwaną (przykładowe kody odpowiedzi w tabeli \ref{tab:Kody-HTTP}). W testach tego typu ważną cechą jest sprawdzenie wartości granicznych (listing \ref{lst:java-UnitTest}). 

Testy integracyjne pozwalają sprawdzić interakcje między komponentami aplikacji dzięki narzędziu MockMVC.
\subsubsection*{Konfiguracja i uruchamianie}
Konfiguracja środowiska odbywa się w pliku \texttt{application.properties}. Definicja plików \texttt{Dockerfile} oraz \texttt{Docker-compose.yml} pozwala na spójne uruchamianie usług w różnych środowiskach.
\subsection*{Przykładowe fragmenty kodu}
\subsubsection*{rejestracja}
Funkcja rejestracji przyjmuje jako argument obiekt klasy User, który posiada poprawnie uzupełnione ciało (login: String, password: String, role: UserRoles). Jeżeli podane dane są poprawne, zostaje wywołana funkcja szyfrująca hasło, a następnie użytkownik jest zapisywany w bazie danych.
\begin{lstlisting}[language={Java}, caption={Rejestracja użytkownika}, label={lst:java-register}]
@PostMapping("/register")
public String register(@Valid @RequestBody User user) {
	user.setPassword(passwordEncoder.encode(user.getPassword()));
	userRepository.insert(user);
	return "Rejestracja udana";
}
\end{lstlisting}
\subsubsection*{Lista użytkowników}
Funkcja \texttt{getUser()} wykorzystuję metodę z repozytorium, która umożliwia pobranie wszystkich dokumentów i zwraca je.
\begin{lstlisting}[language={Java}, caption={Lista użytkowników}, label={lst:java-list}]
@GetMapping("/list")
public Iterable<User> getUser() {
	return userRepository.findAll();
}
\end{lstlisting}
\subsubsection*{Usuwanie użytkownika}
Funkcja \texttt{deleteUser} przyjmuję jako parametr podany w ścieżce identyfikator użytkownika. Następnie jeżeli istnieje usuwa dokument z bazy danych.
\begin{lstlisting}[language={Java}, caption={Usuwanie użytkownika}, label={lst:java-delete}]
@DeleteMapping("/delete/{id}")
public void deleteUser(@PathVariable String id) {
	userRepository.deleteById(id);
}
\end{lstlisting}
\subsubsection*{Edycja użytkownika}
Funkcja \texttt{updateUser} pozwala zmienić login użytkownikowi. Jako parametr przyjmuję identyfikator podany w ścieżce oraz wymagany login. Jeżeli zostały podane poprawne dane aktualizuje dokument w bazie danych.
\begin{lstlisting}[language={Java}, caption={Edycja użytkownika}, label={lst:java-update}]
@PutMapping("/update/{id}")
public void updateUser(@PathVariable String id, @RequestParam String login) {
	User user = userRepository.findById(id).orElseThrow(() -> new RuntimeException("User not found"));
	user.setLogin(login);
	userRepository.save(user);
}
\end{lstlisting}
\subsubsection*{Test Jednostkowy}
Podany poniżej kod testujący służy do sprawdzenia czy aplikacja jest zabezpieczona przed wysyłaniem przelewów, w których podano graniczną wartość, czyli kwotę 0.0 \$. Oczekiwany status odpowiedzi to 4xx, a zwracana odpowiedź to ,,amount: must be greater then 0''
\begin{lstlisting}[language={Java}, caption={Przykładowy test}, label={lst:java-UnitTest}]
@Test
public void createTransferHandlesBorderTransferAmount() throws Exception {
	this.mockMvc.perform(MockMvcRequestBuilders.post("/Transaction/create/transfer")
	.contentType("application/json")
	.content("{\"fromAccountNumber\":\"1234567890122234569012335\",\"toAccountNumber\":\"1234567890122234569012335\",\"amount\":0.0}"))
	.andExpect(status().is4xxClientError())
	.andExpect(content().string("amount: must be greater than 0"));
}
\end{lstlisting}
\section{Frontend – interfejs użytkownika}
Warstwa prezentacji systemu, czyli to z czym użytkownik wchodzi w interakcje nazywamy frontendem. Została zrealizowana przy pomocy biblioteki React oraz narzędzia Vite. Kod źródłowy programu został napisany w TypeScipt. Komunikacja została zapewniona dzięki protokołowi HTTP/REST. Powoduje to niezależność frontendu od backendu.
\subsection*{Struktura projektu}
Projekt jest podzielony na moduły:
\begin{itemize}
	\item Assets - przechowywanie statycznych elementów strony (np. ikony),
	\item Models - Zawiera typy utworzone na potrzeby działania funkcji,
	\item Services - Odpowiada za komunikacje z backendem,
	\item Styles - Przechowuje style w formacie .css,
	\item Context - Obsługuje globalny stan aplikacji,
	\item Components - Zawiera widoki stron i wszystkie elementy takie jak formularze
\end{itemize}
\subsection*{Komunikacja z API}
Komunikacja z backendem odbywa się dzięki bibliotece axios, która odwołuje się do endpointów strony. Poniżej znajduje się przykładowa funkcja wyszukująca konta.
\begin{lstlisting}[caption={Funkcja wyszukująca wszystkie konta użytkownika}, label={lst:TS-service1}]
export const fetchAccounts= async (): Promise<Accounts[]> =>{
	const id = await fetchUserId();
	try {
		const response = await axios.get(`http://localhost:8080/Account/get/${id.id}`, {
			withCredentials: true,
			headers: {
				'Content-Type': 'application/json'
			}
		});
		return await response.data;
	} catch (error: any) {
	throw new Error(error.message || "Wystąpił błąd podczas pobierania kont");
	}
}
\end{lstlisting}

\subsection*{UI}
Interfejs użytkownika jest przejrzysty i łatwy w obsłudze. Wiele elementów na stronie zostało wykorzystane z biblioteki Ant Design, która jest jedną z popularniejszych bibliotek dla React'a. Pozwala ona na tworzenie formularzy, wykresów liniowych, przycisków, ikon i wielu innych przydatnych elementów strony. Użytkownik może personalizować motyw strony (jasny/ciemny), sprawdzić szczegóły profilu lub przejrzeć historię i dane swoich kont bankowych, które ma przypisane do konta.
\subsection*{Stan aplikacji i interakcje}
Stany są jednym z kluczowych funkcji, które wykorzystuje się w aplikacjach opartych o React. Wykorzystują funkcję haków (hook), które pozwalają używać stanów bez konieczności posiadania klasy. Do zarządzania stanami możemy użyć funkcji useState lub useEffect. Funkcja useState pozwala nam ustawić jakiś status podczas działania strony. Prostym przykładem będzie zmiana motywu strony przez użytkownika. Wtedy status zmienia się z light na dark. UseEffect pozwala wykonać jakąś funkcję lub działanie i wpłynąć na to jaki status zostanie ustawiony. Poniżej znajdują się przykłady zastosowań dla obu funkcji.
\begin{lstlisting}[caption={Wykorzystanie stanów}, label={lst:TS-states}]
const [loginData, setLogin] = useState<string | null>(null);
	
useEffect(() => {
	const fetchLoginData = async () => {
		try {
			const users = await fetchUsers();
			setLogin(users.join('\n'));
		} catch (err: any) {
			message.error(err.response?.data?.error || err.message);
		}
	};
	fetchLoginData();
}, []);
\end{lstlisting}
\section{Przepływ danych}
Aplikacja przechowuje dane w postaci dokumentów bazy NoSql MongoDB. Backend komunikuje się z bazą poprzez warstwę repozytoriów i pobiera z niej informacje w formacie JSON, następnie odpowiednio sformatowane dane przesyła odpowiednim endpointem do frontendu, gdy zostanie wywołana odpowiednia metoda.
\subsubsection*{Przykładowy przepływ danych}
Poniżej znajduje się sposób przepływu danych dla funkcji logowania aplikacji:
\begin{enumerate}
	\item Użytkownik odpala formularz na stronie i wprowadza dane.
	\item Frontend wysyła żądanie HTTP do backendu.
	\item Backend przetwarza dane i porównuje je z tymi znajdującymi się w bazie.
	\item Wynik zwraca w formacie JSON.
	\item Frontend aktualizuje widok użytkownika.
\end{enumerate}
\begin{lstlisting}[caption={Przykład przesyłanych danych między frontendem, a backendem}, label={lst:TS-service1}]
[{
	"fromAccountNumber": "3620457673958599558548379",
	"toAccountNumber": "1234567890122234561012335",
	"amount": 50.0,
	"description": "test",
	"transactionId": "68dfb1d240d00a2221f78809"
},
{
	"fromAccountNumber": "3620457673958599558548379",
	"toAccountNumber": "1234567890122234561012335",
	"amount": 150.0,
	"description": "test",
	"transactionId": "68dfb1d740d00a2221f7880b"
}]
\end{lstlisting}
\section{Integracja komponentów}
Komponenty są uruchamiane w oddzielnych kontenerach Dockera. Plik compose.yml pozwala jednocześnie uruchomić bazę danych, backend i frontend przy pomocy jednego polecenia: \textit{docker compose up}.
\begin{lstlisting}[caption={Budowanie obrazu - Dockerfile}, label={lst:Docker-build}]
FROM maven:3.9.10-eclipse-temurin-21
WORKDIR /app
COPY pom.xml .
COPY src ./src
EXPOSE 8080
CMD ["mvn", "spring-boot:run"]
\end{lstlisting}