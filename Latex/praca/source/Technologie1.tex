\chapter{Technologie, z których korzysta aplikacja}
\section{Java}
Java jest obiektowym wieloplatformowym językiem programowania. Jest on najpopularniejszym językiem dla deweloperów oprogramowania \cite{JavaMicrosoft}. Korzysta z maszyn wirtualnych Java (JVM), które mogą zostać zainstalowane na większości komputerów i urządzeń przenośnych. Został on stworzony z myślą "napisz raz, uruchamiaj w dowolnym miejscu". 
\\W projekcie jest odpowiedzialny za logikę biznesową aplikacji, przetwarzanie danych oraz komunikację z bazą danych.
\subsection*{Spring Boot}
Framework Spring Boot pozwala znacząco uprościć proces tworzenia aplikacji webowej w Javie. Eliminuje konieczność ręcznej konfiguracji wielu elementów. Pełni on role fundamentu dla warstwy serwerowej, udostępniając interfejs REST API. Dzięki niemu aplikacja frontendowa może komunikować się z backendem. Składa się z takich rzeczy jak:
\begin{itemize}
	\item Controller - obsługuje żądania HTML
	\item Repository - odpowiada za komunikacje z bazą danych
	\item Config - Obejmuje konfiguracje aplikacji, polityki bezpieczeństwa, ciasteczka
	\item Model - Reprezentuje dane w postaci klas, które są odwzorowaniem tabel w bazie danych.
	\item Resource - Znajdują się w nim właściwości aplikacji, oraz statyczne elementy.
\end{itemize}
Spring Boot wspiera również takie mechanizmy jak Wstrzykiwanie zależności (ang. Dependency Injection) oraz AOP (Aspect-Oriented Programming).
\subsection*{Maven}
W projekcie wykorzystano narzędzie Apache Maven, które pełni role automatyzacji budowy i zarządzania zależnościami. Pozwala zdefiniować wszystkie wymagane biblioteki i frameworki w jednym centralnym pliku pom.xml (ang. Project Object Model). Działa w oparciu o repozytoria, w których przechowywane są biblioteki. Główny plik zawiera takie informacje jak:
\begin{itemize}
	\item groupId, artifactId, version - identyfikatory projektu,
	\item sekcję dependencies - listę bibliotek, które mają zostać automatycznie pobrane,
	\item sekcję build - ustawienia dotyczące budowania aplikacji,
	\item sekcje plugins - narzędzia wspierające proces budowania
\end{itemize}
W przypadku aplikacji opartej na Spring Boot, Maven pobiera również tzw. startery np.:
\begin{itemize}
	\item spring-boot-starter-web - uruchamia aplikacje serwerową (np. Tomcat) i pozwala tworzyć kontrolery REST
	\item spring-boot-starter-data-mongodb - Umożliwia korzystanie z bazy mongo
	\item spring-boot-starter-security - Dodaje mechanizmy autoryzacji i uwierzytelniania
	\item spring-boot-starter-test - pozwala na testowanie aplikacji
\end{itemize}
Cała konfiguracja środowiska sprowadza się do utworzenia odpowiedniego pliku z zależnościami, a Spring Boot automatycznie na jego podstawie konfiguruje je przy uruchomieniu.

\section{React}
React to jedna z bibliotek JavaScript służąca do tworzenia interfejsów użytkownika (ang. User Interface) w aplikacjach webowych. Jest oparta na koncepcji komponentów, które można łatwo ze sobą łączyć i w czytelny sposób wyświetlać na stronie. React pozwala renderować widoki na podstawie mechanizmu Virtual DOM (ang. Document Object Model). Obiekt DOM umożliwia dostęp do struktury strony w celu jej modyfikacji. W przypadku modelu wirtualnego minimalizuje operacje na drzewie rzeczywistym. 
\subsection*{Vite}
W celu usprawnienia procesu tworzenia aplikacji zastosowano narzędzie Vite, które pełni rolę bundlera, czyli łączy ze sobą wiele plików m.in. kody źródłowe i zależności. Rozwiązanie to oferuje szybkie uruchamianie środowiska, optymalizacja kodu i co najważniejsze w usprawnieniu pracy poprzez przeładowanie kodu na bieżąco (ang. Hot Module Replacement), który pozwala wyświetlać zmiany bez konieczności ponownego budowania całej aplikacji. 
\subsection*{TypeScript}
Komponenty oraz wszystkie składowe projektu są przygotowane w języku TypeScript, który jest nadzbiorem języka JavaScript. Wprowadza on statyczne typowanie, znaczy to że typy są określone i sprawdzane w czasie kompilacji. Pozwala wprowadzać typy dla zmiennych, funkcji i obiektów. Dzięki temu kod jest bardziej czytelny i zrozumiały dla developera. 
\newline

Połączenie Reacta, Vite oraz TypeScript zapewnia wydajne i przyjemne środowisko dla pracy programisty, lepszą kontrole nad danymi i wydajność pracy. Technologie te zostały wybrane również dla bogatego wyboru bibliotek wspieranych przez react. Jedną z nich jest biblioteka AntD, która zawiera wstępnie przygotowane do użycia na stronie komponenty. 

\section{MongoDB}
MongoDB to nierelacyjna baza typu NoSQL. W przeciwieństwie do klasycznych baz relacyjnych MongoDB nie korzysta z tabel, lecz przechowuje dane w postaci dokumentów BSON (Binary JSON), które strukturą przypominają obiekty JSON.
Jeden dokument może mieć wiele różnych pól, co pozwala bardziej elastycznie modelować dane. Baza ta korzysta z kolekcji (ang. collections), które są odpowiednikami tabel w klasycznych bazach relacyjnych. Każdy wpis w bazie posiada własny unikatowy identyfikator \_id. Wspiera ona wiele operacji, takich jak sortowanie, filtrowanie, czy grupowanie. Zaimplementowanie ich w kodzie programu jest proste i wystarczy do tego odpowiednio utworzone repozytorium z odpowiednimi operacjami (lst. \ref{lst:java-repo}). Połączenie bazy w aplikacji Java odbywa się za pomocą modułu Spring Data MongoDB. To właśnie ten moduł odpowiada za możliwość tworzenia repozytorium.
\begin{lstlisting}[language={Java}, caption={Przykładowe repozytorium w Javie}, label={lst:java-repo}]
@RepositoryRestResource(collectionResourceRel = "users", path = "users")
public interface UserRepository extends MongoRepository<User, String> {
	Optional<User> findByLogin(String login);
	List<User> findByLoginContainingIgnoreCase(String loginPart);	
}	
\end{lstlisting}
Powyższy fragment kodu pozwala znaleźć użytkownika po loginie lub wyszukać wszystkich użytkowników zawierających podany ciąg znaków w swojej nazwie.
\begin{lstlisting}[caption={Przykladowy dokument z kolekcji}, label={lst:MongoDB-doc}]
{
	"id": "68d56ad95a544e07c8ebaa54",
	"login": "ADMIN",
	"password": "$2a$10$.itTB4jFiMBPdZSDebVE4Obtl8FpDiT7CHovqCtq8dcUnFMoe1gem",
	"role": "ADMIN"
}
\end{lstlisting}

\section{Postman}
Postman to narzędzie pozwalające na testowanie i analizę interfejsów API. Umożliwia wysyłanie żądań HTTP (GET, POST, PUT, DELETE) do serwera i sprawdzanie zwracanych odpowiedzi. Pozwala to na szybkie i prostą weryfikację poprawności działania endpointów. Umożliwia on tworzenie kolekcji z zapytaniami, które można używać wiele razy bez konieczności zapisu ich po każdym użyciu. Posiada też możliwość zapisywania ciasteczek, tworzenia własnych nagłówków lub przesyłanego ciała do funkcji. 
\section{Docker} 
Docker pozwala na konteneryzacje aplikacji. Jego głównym elementem jest silnik (Docker Engine), który działa w formie usługi. Umożliwia on budowanie, uruchamianie i zarządzanie kontenerami. W środowisku Dockera możemy tworzyć wirtualne obrazy, które posłużą nam do wykonania jakiegoś zadania. Na bazie obrazów możemy tworzyć kontenery. Plusem takiego rozwiązania jest możliwość określenia wersji wszystkich użytych technologii, jak również perspektywa odpalenia naszego programu na dowolnym urządzeniu (przenośność), które posiada zainstalowany docker. Projekty bardziej złożone (korzystające z wielu usług) możemy połączyć za pomocą pliku compose.yml, w którym opisujemy zależności pomiędzy danymi kontenerami np.: określamy sieć w jakiej mają się znajdować, woluminy z jakich mają korzystać lub elementy od których są zależne.
\section{Git}
System kontroli wersji to jedna z najważniejszych rzeczy podczas pisania dowolnej aplikacji. Najczęściej używanym rozwiązaniem jest git. Pozwala on na tworzenie lokalnych lub zdalnych repozytoriów do których zapisywane są zmiany w projekcie (commit). W dowolnym momencie możemy śledzić modyfikacje plików lub powrócić do wcześniejszej wersji projektu (rollback). Aby przesłać zmiany do zdalnego repozytorium na stronie trzeciej (np.: github, bitbucket), należy wykorzystać opcję (push), natomiast pobranie wymaga operacji (pull). 

Git umożliwia również pracę w grupach dzięki systemowi tworzenia gałęzi (branches), które pozwalają na równoległą prace całego zespołu. Zakończone części kodu mogą być scalone (marge) z główną gałęzią.
