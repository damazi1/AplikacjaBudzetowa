\chapter{Podsumowanie i wnioski }

Celem niniejszej pracy było zaprojektowanie i implementacja aplikacji webowej wspomagającej zarządzanie finansami. Aplikacja powstała w oparciu o architekturę klient-serwer i posiada takie funkcjonalności jak:
\begin{itemize}
	\item Rejestrowanie i logowanie użytkownika,
	\item Tworzenie różnego rodzaju transakcji,
	\item Sprawdzanie szczegółów profilu lub kont,
	\item Analiza danych finansowych przy pomocy wykresów,
	\item Komunikacja między frontendem, a backendem przy pomocy REST API,
	\item Przechowywanie danych w bazie nierelacyjnej MongoDB.
\end{itemize}
Aplikacja umożliwia elastyczne zarządzanie danymi, dzięki bazie MongoDB. Jej uniwersalny interfejs jest prosty w obsłudze co ułatwia poruszanie się po stronie dla każdej grupy wiekowej. Projekt pozwoli użytkownikowi w łatwiejszy sposób robić takie rzeczy jak:
\begin{itemize}
	\item Zarządzać własnym budżetem domowym,
	\item Śledzić swoje wydatki,
	\item Planować przyszłe płatności,
	\item Wykonywać kroki w celu inwestycji środków.
\end{itemize}
\section*{Wnioski}
Nowoczesne technologie webowe pozwalają na tworzenie skalowalnych i łatwych w rozwoju aplikacji. Architektura klient-serwer sprawdziła się w rozdzieleniu warstwy frontendu od backendu.

W przyszłości aplikację będzie można rozszerzyć o:
\begin{itemize}
	\item Funkcje prognozowania wydatków z wykorzystaniem algorytmów uczenia maszynowego,
	\item Możliwość eksportu danych do formatu PDF lub CSV.
\end{itemize}

Projekt umożliwił rozwinięcie umiejętności analitycznych oraz technicznych związanych z tworzeniem aplikacji webowych.

\section*{Co osiągnąłem}
Podczas realizacji projektu osiągnięto wiele rezultatów. Stworzono kompletną aplikację webową umożliwiającą zarządzanie budżetem domowym, w której połączono wiele nowoczesnych rozwiązań technologicznych. Do najważniejszych osiągnięć należą:
\begin{itemize}
	\item Wykorzystanie architektury klient-serwer,
	\item Interfejs oparty o technologię React,
	\item Opracowanie logiki biznesowej przy pomocy Spring Boot,
	\item Uzyskanie płynnie działającego przepływu danych w bazie MongoDB,
	\item Wdrożenie środowiska do uruchamiania w oparciu o Docker,
	\item Opracowanie testów funkcjonalnych przy użyciu narzędzia Postman,
	\item Umożliwienie dalszej rozbudowy systemu.
\end{itemize}

Realizacja powyższych elementów pozwoliła na stworzenie w pełni działającego systemu, który spełnia założenia projektowe. 

W trakcie tworzenia projektu autor rozwinął swoje umiejętności w zakresie programowania w językach TypeScript i Java, projektowania aplikacji webowych oraz pracy z bazami typu NoSQL. Zdobyte doświadczenie obejmuje również prace z narzędziami takimi jak Git, Docker oraz Postman.