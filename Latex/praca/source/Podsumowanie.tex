\chapter{Podsumowanie i~wnioski}
W teraźniejszych czasach ludzie coraz częściej korzystają z~portfeli elektronicznych. Pozwalają one na proste zarządzanie finansami bez konieczności wychodzenia z~domu. Dzięki nim można kategoryzować swoje wydatki, przeglądać szczegółowe statystyki, oszczędzać pieniądze i~usprawnić zarządzanie finansami. 

Celem niniejszej pracy było zaprojektowanie i~implementacja aplikacji webowej wspomagającej zarządzanie finansami osobistymi. Aplikacja została utworzona w~oparciu o~różne technologię i~narzędzia.

System powstał i~w pełni spełnia założone cele. Umożliwia dodawanie i~usuwanie środków, śledzenie wydatków, kategoryzowanie i~ogólne zarządzanie budżetem. Interfejs użytkownika pozwala intuicyjnie poruszać się po elementach aplikacji, a~funkcje i~metody opisane po stronie serwera, dają możliwość na wydajne przetwarzanie danych. 
\section*{Wnioski}
Nowoczesne technologie pozwalają na tworzenie skalowalnych i~łatwych w~rozwoju aplikacji. Narzędzia programistyczne są niezbędne w~procesie tworzenia oprogramowania. 

Odpowiednie przygotowanie architektury aplikacji pozwala na implementację klas, metod i~funkcji, które zostały zaplanowane na początku tworzenia projektu. 
W procesie wytwarzania oprogramowania może okazać się, że nie wszystkie zaplanowane rozwiązania będą odpowiednie dla danego systemu.

Język Java wraz z~frameworkiem Spring jest idealny do tworzenia aplikacji webowej. Zawiera wiele bibliotek, które czynią to rozwiązanie jednym z~najlepszych. Testowanie oprogramowania pozwala ograniczyć ilość błędów systemu. 

Biblioteka React wraz z~narzędziami deweloperskimi pozwala na sprawny rozwój interfejsu użytkownika. Dostępność wielu gotowych komponentów sprawia, że warstwa prezentacji jest przejrzysta i~czytelna. 

Środowiska deweloperskie pozwalają na szybkie sprawdzanie wprowadzanych zmian dzięki funkcji przeładowania na gorąco. 

Nierelacyjne bazy danych pozwalają na dynamiczną edycję struktury dokumentów, bez konieczności edycji istniejących rekordów.

Konteneryzacja umożliwia tworzenie bezpiecznych środowisk, w~których może działać system.
Kontrola wersji aplikacji pozwala na bezpieczny rozwój projektu, dzięki udostępnionej możliwości zapisywania zmian.

\section*{Osiągnięcia autora}
Podczas tworzenia projektu autor rozwinął swoje umiejętności techniczne oraz analityczne w~zakresie tworzenia aplikacji webowych. Poszerzył zakres wiedzy w~dziedzinie informatyki i~inżynierii oprogramowania. Opracował takie zagadnienia jak:
\begin{itemize}
	\item Implementacja architektury warstwowej,
	\item Obsługa komunikacji warstw,
	\item Wzorce DTO, 
	\item Globalna obsługa wyjątków,
	\item Bezpieczeństwo bezstanowe,
	\item Implementacja tokenów JWT,
	\item Testowanie aplikacji z~wykorzystaniem mocków,
	\item Wizualizacja danych analitycznych,
	\item Konteneryzacja środowiska,
	\item Modelowanie danych w~bazie NoSql,
	\item Implementacje interfejsu użytkownika z~wykorzystaniem biblioteki React,
	\item Wykorzystanie frameworka Spring,
	\item Opracowanie diagramów UML,
	\item Język TypeScript,
\end{itemize}   
\section*{Przyszły rozwój aplikacji}
W przyszłości aplikację będzie można rozszerzyć o~takie funkcjonalności jak 
\begin{itemize}
	\item Prognozowania wydatków z~wykorzystaniem algorytmów uczenia maszynowego.
	\item Dodawanie kart płatniczych.
	\item Możliwość ustawienia awataru użytkownika.
	\item Utworzenie raportów z~danymi w~formacie PDF.
	\item Konwersja danych do formatu XML.
	\item Rozszerzenie tłumaczeń strony.
	\item Zwiększony wybór walut.
	\item Dodanie api obsługi realnych kont bankowych.
\end{itemize}