\section{angular}
Ważne komendy:
\begin{itemize}
	\item Tworzenie komponentów
	\\ng g c [nazwa]
	\\ng generate component [nazwa]
	\\ng generate component sciezka/[nazwa]
	\item sprawdzenie co wykona komenda
	\\ --dry-run
	\item DataBinding - tworzenie zmiennych w tekst tak jakby
	\\ {{ zmienna }} - tradycyjny sposob
	\\ {{ funkcja () }} - signal
	\item Routing - bardzo ważna rzecz
	\\ służy do tworzenia strony na jednej stronie
	\\ Wszystko w app.routes
	\\ W templatce podajemy  <routing> i wtedy w zaleznosci od adresu przenosi nas na odpowiednią strone
	\\item Tworzenie klas 
	\\ng g class [nazwa]
	\\item Tworzenie serwisów
	\\ng g s [nazwa]
	\item łączenie się z serverem - Tworzymy serwis o jakiejś nazwie następnie w środku serwisu umieczsamy adress http z którego skorzystamy oraz prywatny adres klienta
	\begin{lstlisting}[language=Java, caption=Łączenie z serwerem]
@Injectable({
 providedIn: 'root'
})
export class ServiceName {
  //bazowy adres serwera
  private baseUrl = 'http://localhost:8080/list';

  constructor(private  httpClient: HttpClient) { }
	
  getUsersList(): Observable<User[]>{
    return this.httpClient.get<User[]>(`$\textdollar${this.baseUrl}`);
  }
}
	\end{lstlisting}
	\item Wstrzykiwanie zależności jest możliwe poprzez polecenie @Injactible oraz narzędzia dostarczania usług (providers). Tworzymy usługe w niej właśnie znajduje się możliwość wstrzyknięcia tejże usługi. Następnie w jakimś komponencie może zostać wstrzyknięta usługa.
	\begin{lstlisting}[language=Java, caption=Wstrzyknięcie serwisu]
export class NazwaKomponentu implements OnInit {
 constructor(private userService: UserService) {
  //Wstrzykiwanie uslugi
 }
 //Dalsza przykladowa funkcjonalnosc 
 ngOnInit() {
  this.getUsers();
 }
 getUsers() {
  this.userService.getUsersList().subscribe(data => {
   this.users = data;
  });
}
	\end{lstlisting}
	
\end{itemize}