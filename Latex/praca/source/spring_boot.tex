\section{Spring Framework \cite{HK}}
Jest to struktura, która dostarcza funkcjonalność dla aplikacji. Odpowiada ona między innymi za: 
\begin{itemize}
    \item wstrzykiwanie zależności (Dependency Injection)
    \item Obsługe zdarzeń (Events)
    \item Zarządzanie zasobami (Resources)
    \item Walidacje (Validation)
    \item Wiązanie danych (Data binding)
    \item Konwersje typów (Type conversion)
    \item SpEL
    \item AOP
\end{itemize}
Spring jest bardzo popularnym frameworkiem do budowania aplikacji webowych.
\\Jego popularność zawdzięcza takim rzeczą jak
\begin{itemize}
    \item łatwe wstawianie zależności (Dependency Injection)
    \item dobrze zintegrowany z innymi frameworkami javy takimi jak JPA/Hibernate
    \item Posiada framework MVC do budowania aplikacji webowych
\end{itemize}
\subsection{Spring boot}
\begin{itemize}
    \item \textbf{Starter} pozwala skorzystać z wcześniej skonfigurowanych zależności 
    \item przykłady starterów
    \begin{itemize}
        \item spring-boot-starter-data-jpa
        \item spring-boot-starter-web
    \end{itemize}
    \item \textbf{Automatyczna konfiguracja} - spring boot jest wyposażony w narzędzie, które na bazie zależności \textbf{jar}, które dodaliśmy do projektu  próbuje skonfigurować projekt
\end{itemize}

\subsection{Spring MVC}
Jest on zaawansowanym frameworkiem webowym. Składa się on z takich elementów jak:
\begin{itemize}
    \item Dyspozytor Serwletu (DispatcherServlet)
    \item Mapowanie żądań (Request Mapping)
    \item Metody obsługi żądań (Handler methods)
    \item Obsługa wyjątków (Exepcitons handling)
\end{itemize}