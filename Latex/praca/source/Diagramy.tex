\chapter{Architektura aplikacji}
W rozdziale przedstawiono wymagania funkcjonalne i niefunkcjonalne. Zilustrowano również klasy oraz funkcje programu na diagramach języka UML.
\section{Wymagania funkcjonalne}
Wymagania funkcjonalne określają zachowanie systemu i jego reakcji na określone zadania \cite{Wymagania}. Stanowią one podstawę dla wykonawcy oprogramowania do opracowania odpowiednich metod oraz funkcjonalności. Precyzyjne zdefiniowanie wymagań z perspektywy biznesowej pozwala na dopasowanie tworzonych rozwiązań do potrzeb organizacji oraz celów strategicznych projektu. Takie rozwiązanie umożliwia wdrożenie systemu zgodnego z oczekiwaniami zleceniodawcy.

\begin{longtable}{|c|p{5cm}|p{8cm}|}
	\caption{Wymagania funkcjonalne systemu do zarządzania budżetem domowym}
	\label{tab:wymagania_funkcjonalne} \\
	\hline
	\textbf{Nr} & \textbf{Wymaganie funkcjonalne} & \textbf{Opis} \\ \hline
	
	1. & Rejestracja użytkownika & System umożliwia utworzenie konta użytkownika z danymi logowania. \\ \hline
	2. & Logowanie do systemu & System weryfikuje dane użytkownika i przyznaje ciasteczko umożliwiające dostęp do aplikacji. \\ \hline
	3. & Wylogowanie & Użytkownik usuwa swoje ciasteczko \\ \hline
	4. & Sprawdzenie dostępnych środków & Użytkownik może sprawdzić balans w portfelach, kontach bankowych i łączny balans. \\ \hline
	5. & Tworzenie transakcji  & Użytkownik może utworzyć wpłatę lub wypłatę z konta bankowego i portfelu. \\ \hline
	6. & Historia transakcji & Aplikacja wyświetla listę wszystkich transakcji. \\ \hline
	7. & Tworzenie przelewu & Użytkownik może przelać finanse na inne konto bankowe. \\ \hline
	8. & Podsumowanie finansów & System wyświetla przychody i wydatki w określonym czasie. \\ \hline
	9. & Kategoryzowanie transakcji & Użytkownik podaje kategorię dla swoich transakcji w portfelu. \\ \hline
	10. & Prezentacja wydatków na wykresie & System wyświetla dane o transakcjach w postaci wykresów. \\ \hline
	11. & Kantor walut & System przelicza finanse do innych walut. \\ \hline
	12. & Uzupełnienie informacji o użytkowniku & Użytkownik może przypisać dane kontaktowe do konta. \\ \hline
\end{longtable}

\section{Wymagania niefunkcjonalne}
Wymaganiami niefunkcjonalnymi nazywamy wszystkie potrzeby, które nie odnoszą się bezpośrednio do funkcjonalności produktu, lecz określają jego właściwości jakościowe \cite{Wymagania}. Dotyczą one m.in. takich zagadnień jak czas reakcji systemu, termin dostarczenia produktu, wyglądu interfejsu użytkownika, czy bezpieczeństwa. Stanowią kluczowy element dla wykonawcy oprogramowania, przy projektowaniu architektury, doborze technologii oraz planowaniu procesu wdrożenia. Mają istotne znaczenie z perspektywy biznesowej, ponieważ wpływają na satysfakcję użytkowników, konkurencyjność systemu oraz koszty jego utrzymania. Dla zamawiających gwarantują odpowiednią jakość, stabilność i wydajność oprogramowania. 

\begin{longtable}{|c|p{3.1cm}|p{9.9cm}|}
		\caption{Wymagania niefunkcjonalne systemu do zarządzania budżetem domowym}
	\label{tab:wymagania_niefunkcjonalne} \\
	\hline
	\textbf{Nr} & \textbf{Wymaganie niefunkcjonalne} & \textbf{Opis} \\ \hline
	
	1. & Niezawodność & System w krótkim czasie odzyskuje pełną funkcjonalność w przypadku wystąpienia błędu. \\ \hline
	2. & Dostępność & System ma być dostępna przez 99.9\% czasu działania serwera.\\ \hline
	3. & Wydajność & System reaguje na żądanie w czasie krótszym niż 1 sekunda. Z systemu jednocześnie może korzystać więcej niż jeden użytkownik.\\ \hline
	4. & Bezpieczeństwo & System zapewnia, że dane są dostępne tylko dla osób upoważnionych. Autoryzacja użytkowników odbywa się poprzez podanie odpowiedniego loginu oraz hasła. Hasła muszą być przechowywane w formie zaszyfrowanej. \\ \hline
	5. & Wdrożenie & Aplikacja serwerowa musi być uruchomiona w środowisku JVM oraz korzystać z wersji Java 21, Warstwa prezentacji uruchamiana w środowisku NodeJs z wykorzystaniem 9 wersji biblioteki React. Dane przechowywane w postaci dokumentów bazy danych MongoDB w wersji 8, Aplikacja uruchomiana jest w środowisku systemu operacyjnego Windows 11 oraz Linux \\ \hline
	
\end{longtable}


\section{Diagramy UML}
Jednym z kluczowych etapów w procesie tworzenia oprogramowanie jest wizualizacja systemu \cite{UML}. Język UML (ang. \textit{Unified Modeling Language}), czyli ujednolicony język modelowania nie jest językiem programowani, chociaż może uprościć proces tworzenia aplikacji generując kod na podstawie diagramów. UML definiuje dwie podstawowe składowe: notację elementów na diagramach oraz ich semantykę \cite{UML1}. W artykule T. Sobestańczyk diagram opisany jest słowami ,,Diagram jest schematem przedstawiającym zbiór bytów''.
\subsection*{Tworzenie diagramów}
Diagramy UML można utworzyć graficznie, na przykład przez strone draw.io, bądź tekstowo odpowiednim językiem takim jak PlantUML lub mermaid. Można też skorzystać z narzędzi takich jak enterprise architect, który posiada dodatkowe opcję takie jak tworzenie kodu programu na bazie diagramów. 
\section{Diagramy przypadków użycia}
Diagram przypadków użycia (ang. \textit{Use Case Diagram}) prezentuje usługi, które system świadczy aktorom, bez wskazywania rozwiązań technologicznych \cite{UML1}. Stanowi on podstawę do modelowania szczegółowych części systemu.
\begin{figure}[H]
	\centering
	\includegraphics[width=1\textwidth]{images/PlantUML_useCase.png}
	\caption{Diagram przypadków użycia}
	\label{fig:UseCase}
\end{figure}
\section{Diagramy sekwencji}
Diagram sekwencji (ang. \textit{sequence diagram}) jest uzupełnieniem diagramu klas, który reprezentuje statyczną strukturę systemu. Diagram sekwencji w dynamiczny sposób przedstawia zachowanie klas, interfejsów oraz możliwe zastosowanie metod.
\begin{figure}[H]
	\centering
	\includegraphics[width=0.9\textwidth]{images/MermaidSeqDiag}
	\caption{Diagram sekwencji - Nowa transakcja portfela}
	\label{fig:Seq1}
\end{figure}
\section{Diagram klas}
Przedstawia klasy oraz zależności między nimi:
\begin{figure}[H]
	\centering
	\includegraphics[height=1\textheight, keepaspectratio]{images/testDiag}
	\caption{Diagram klas}
	\label{fig:mermaidclassdiag}
\end{figure}


