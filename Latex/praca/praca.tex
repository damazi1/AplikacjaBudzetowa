\documentclass[a4paper,11pt,twoside]{mgr}

%**************************************************************************
% Dane do strony tytułowej
%

% Autor
\autor{Dawid Ziora}

% Rodzaj pracy - wpisać LICENCJACKA, INŻYNIERSKA lub MAGISTERSKA
\rodzajPracy{INŻYNIERSKA}

% Tytuł pracy magisterskiej/inżynierskiej
\tytul{Aplikacja klient-serwer}

% Tytuł pracy magisterskiej w języku angielskim
\tytulAng{client-server application}

% Promotor
\promotor{dr inż. Bartosz Kowalczyk}

% Rok
\rok{2024}

% Kierunek
\kierunek{Informatyka}

% Studia stacjonarne lub niestacjonarne (wpisać jakie)
\studia{stacjonarne}

% Poziom studiów wpisać I lub II
\poziomStudiow{I}

% Numer albumu
\numerAlbumu{136700}

%
%**************************************************************************

% Styl dla wtrąceń anglojęzycznych
\newcommand{\eng}[1]{(\emph{#1})}
%\renewcommand{\baselinestretch}{1}
\renewcommand\lstlistlistingname{Spis listingów}

\begin{document}

\pagestyle{empty}
\stronaTytulowa
\cleardoublepage

\tableofcontents

\pagestyle{fancy}

\addcontentsline{toc}{chapter}{Wstęp}
\chapter*{Wstęp}
Aplikacje webowe są bardzo powszechnie używanymi aplikacjami naszych czasów. Coraz częściej chcemy aby na naszych urządzeniach nie pojawiało się wiele różnego rodzaju aplikacji, tylko by wszystko mogło zostać upakowane w przeglądarke internetową z której prosto i szybko uzyskamy dostęp do programów i aplikacji których potrzebujemy. 
W tej pracy postaram skupić się na aplikacji typu fintech, czyli aplikacji związanej z usługami finansowymi.
\section*{cel pracy}
Celem mojej pracy będzie uzyskanie działającej aplikacji bankowej w której klient będzie porozumiewał się z serwerem przy pomocy różnego rodzaju zapytań.
\section*{zakres pracy}
Językiem z którego skorzystam podczas pisania pracy będzie Java oraz Spring (Do logiki serwerowej - Backendu), baza danych MongoDB do obsługi bazy danych oraz Angular i npm do obsługi frontendu.
\section*{opisówka}
\chapter{Technologie, z których korzysta aplikacjia}
\section{Spring Framework \cite{HK}}
Jest to struktura, która dostarcza funkcjonalność dla aplikacji. Odpowiada ona między innymi za: 
\begin{itemize}
    \item wstrzykiwanie zależności (Dependency Injection)
    \item Obsługe zdarzeń (Events)
    \item Zarządzanie zasobami (Resources)
    \item Walidacje (Validation)
    \item Wiązanie danych (Data binding)
    \item Konwersje typów (Type conversion)
    \item SpEL
    \item AOP
\end{itemize}
Spring jest bardzo popularnym frameworkiem do budowania aplikacji webowych.
\\Jego popularność zawdzięcza takim rzeczą jak
\begin{itemize}
    \item łatwe wstawianie zależności (Dependency Injection)
    \item dobrze zintegrowany z innymi frameworkami javy takimi jak JPA/Hibernate
    \item Posiada framework MVC do budowania aplikacji webowych
\end{itemize}
\subsection{Spring boot}
\begin{itemize}
    \item \textbf{Starter} pozwala skorzystać z wcześniej skonfigurowanych zależności 
    \item przykłady starterów
    \begin{itemize}
        \item spring-boot-starter-data-jpa
        \item spring-boot-starter-web
    \end{itemize}
    \item \textbf{Automatyczna konfiguracja} - spring boot jest wyposażony w narzędzie, które na bazie zależności \textbf{jar}, które dodaliśmy do projektu  próbuje skonfigurować projekt
\end{itemize}

\subsection{Spring MVC}
Jest on zaawansowanym frameworkiem webowym. Składa się on z takich elementów jak:
\begin{itemize}
    \item Dyspozytor Serwletu (DispatcherServlet)
    \item Mapowanie żądań (Request Mapping)
    \item Metody obsługi żądań (Handler methods)
    \item Obsługa wyjątków (Exepcitons handling)
\end{itemize}
\section{angular}
Ważne komendy:
\begin{itemize}
	\item Tworzenie komponentów
	\\ng g c [nazwa]
	\\ng generate component [nazwa]
	\\ng generate component sciezka/[nazwa]
	\item sprawdzenie co wykona komenda
	\\ --dry-run
	\item DataBinding - tworzenie zmiennych w tekst tak jakby
	\\ {{ zmienna }} - tradycyjny sposob
	\\ {{ funkcja () }} - signal
	\item Routing - bardzo ważna rzecz
	\\ służy do tworzenia strony na jednej stronie
	\\ Wszystko w app.routes
	\\ W templatce podajemy  <routing> i wtedy w zaleznosci od adresu przenosi nas na odpowiednią strone
	\\item Tworzenie klas 
	\\ng g class [nazwa]
	\\item Tworzenie serwisów
	\\ng g s [nazwa]
	\item łączenie się z serverem - Tworzymy serwis o jakiejś nazwie następnie w środku serwisu umieczsamy adress http z którego skorzystamy oraz prywatny adres klienta
	\begin{lstlisting}[language=Java, caption=Łączenie z serwerem]
@Injectable({
 providedIn: 'root'
})
export class ServiceName {
  //bazowy adres serwera
  private baseUrl = 'http://localhost:8080/list';

  constructor(private  httpClient: HttpClient) { }
	
  getUsersList(): Observable<User[]>{
    return this.httpClient.get<User[]>(`$\textdollar${this.baseUrl}`);
  }
}
	\end{lstlisting}
	\item Wstrzykiwanie zależności jest możliwe poprzez polecenie @Injactible oraz narzędzia dostarczania usług (providers). Tworzymy usługe w niej właśnie znajduje się możliwość wstrzyknięcia tejże usługi. Następnie w jakimś komponencie może zostać wstrzyknięta usługa.
	\begin{lstlisting}[language=Java, caption=Wstrzyknięcie serwisu]
export class NazwaKomponentu implements OnInit {
 constructor(private userService: UserService) {
  //Wstrzykiwanie uslugi
 }
 //Dalsza przykladowa funkcjonalnosc (uzycie uslugi)
 ngOnInit() {
  this.getUsers();
 }
 getUsers() {
  this.userService.getUsersList().subscribe(data => {
   this.users = data;
  });
}
	\end{lstlisting}
	\item obietnice (promise) - to taki asynchroniczny element, który pozwala odroczyć wczytanie pewnej zależności w czasie i wczytać ją dopiero wtedy gdy będzie potrzebna.
	\begin{lstlisting}[language=Java, caption=Promise]
{path: 'home',
 loadComponent: async () => {
  const m = await import('./home/home.component');
  return m.HomeComponent;}},
	\end{lstlisting}
	Wczytuje komponent home tylko jeśli zostanie wywołana ścieżka home. (Przynajmniej na ten moment tak to rozumiem).
	
\end{itemize}
\section{MongoDB}
\chapter{Wymagania i diagramy}
\section{wymagania funkcjonalne}
\section{wymagania niefunkcjonalne}
\section{Diagramy przypadków użycia}
\section{Diagramy sekwencji}
\chapter{Kod, funkcje, itd. CORE}
\chapter{podsumowanie oraz wnioski}
\addcontentsline{toc}{chapter}{Zakończenie}
\addcontentsline{toc}{section}{Co osiągnąłem}
\section*{Co osiągnąłem}
\addcontentsline{toc}{chapter}{Streszczenie}

\bibliography{zrodla}

\listoffigures

\listoftables

\addcontentsline{toc}{chapter}{\lstlistlistingname}
\end{document}
